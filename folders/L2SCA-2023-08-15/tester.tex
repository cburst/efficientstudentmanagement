\documentclass{article}
\usepackage{ulem}
\usepackage{xcolor}
\usepackage{geometry}
\geometry{a4paper, margin=1in}
\sloppy
\hyphenpenalty=10000
\tolerance=1000
\begin{document}
\indent \textbf{Analysis notes:} \newline
\indent This PDF file contains your text underlined using the Stanford Parser to emphasize the presence of complex nominals, which in turn are associated with syntactic complexity. Complex nominals are essentially sophisticated names that include several words or phrases as part of the name. Try to use complex nominals often, but as efficiently as possible.
\newline
\newline
\indent \textbf{Contact info:} \newline
\indent \begin{color}{teal} richard.rose@yonsei.ac.kr \end{color} \newline
\newline
\newline
\indent \textbf{Your text:} 
\indent


\uline{The operation of \uline{\uline{several university course terms} as \uline{arms of \uline{a randomized controlled trial}}}} may also constitute stretching \uline{the meaning of \uline{the RCT concept}}. In particular, there is a possibility that \uline{students in successive terms} heard about \uline{\uline{the personal experiences} of students} in \uline{\uline{previous terms}} (\uline{either positive or negative}), \uline{which influenced their decision to enroll in \uline{the course in later terms}}. \uline{This potential selection bias} could lead to \uline{contamination of \uline{the later terms}}, weakening \uline{the strength of \uline{clausal inferences generated by \uline{this intervention research project}}}. Nonetheless, educator researchers have a responsibility to construct \uline{\uline{the most valid and reliable research project} they can}, given the resources available to them. Given \uline{the difficulty of participant recruitment}, \uline{this RCT structure} appears to constitute \uline{a suitable and accessible option}. 

\end{document}